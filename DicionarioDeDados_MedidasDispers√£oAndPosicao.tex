% Options for packages loaded elsewhere
\PassOptionsToPackage{unicode}{hyperref}
\PassOptionsToPackage{hyphens}{url}
%
\documentclass[
]{article}
\usepackage{amsmath,amssymb}
\usepackage{lmodern}
\usepackage{ifxetex,ifluatex}
\ifnum 0\ifxetex 1\fi\ifluatex 1\fi=0 % if pdftex
  \usepackage[T1]{fontenc}
  \usepackage[utf8]{inputenc}
  \usepackage{textcomp} % provide euro and other symbols
\else % if luatex or xetex
  \usepackage{unicode-math}
  \defaultfontfeatures{Scale=MatchLowercase}
  \defaultfontfeatures[\rmfamily]{Ligatures=TeX,Scale=1}
\fi
% Use upquote if available, for straight quotes in verbatim environments
\IfFileExists{upquote.sty}{\usepackage{upquote}}{}
\IfFileExists{microtype.sty}{% use microtype if available
  \usepackage[]{microtype}
  \UseMicrotypeSet[protrusion]{basicmath} % disable protrusion for tt fonts
}{}
\makeatletter
\@ifundefined{KOMAClassName}{% if non-KOMA class
  \IfFileExists{parskip.sty}{%
    \usepackage{parskip}
  }{% else
    \setlength{\parindent}{0pt}
    \setlength{\parskip}{6pt plus 2pt minus 1pt}}
}{% if KOMA class
  \KOMAoptions{parskip=half}}
\makeatother
\usepackage{xcolor}
\IfFileExists{xurl.sty}{\usepackage{xurl}}{} % add URL line breaks if available
\IfFileExists{bookmark.sty}{\usepackage{bookmark}}{\usepackage{hyperref}}
\hypersetup{
  pdftitle={Atividade trilha 1 - Analise estátisticaca},
  hidelinks,
  pdfcreator={LaTeX via pandoc}}
\urlstyle{same} % disable monospaced font for URLs
\usepackage[margin=1in]{geometry}
\usepackage{color}
\usepackage{fancyvrb}
\newcommand{\VerbBar}{|}
\newcommand{\VERB}{\Verb[commandchars=\\\{\}]}
\DefineVerbatimEnvironment{Highlighting}{Verbatim}{commandchars=\\\{\}}
% Add ',fontsize=\small' for more characters per line
\usepackage{framed}
\definecolor{shadecolor}{RGB}{248,248,248}
\newenvironment{Shaded}{\begin{snugshade}}{\end{snugshade}}
\newcommand{\AlertTok}[1]{\textcolor[rgb]{0.94,0.16,0.16}{#1}}
\newcommand{\AnnotationTok}[1]{\textcolor[rgb]{0.56,0.35,0.01}{\textbf{\textit{#1}}}}
\newcommand{\AttributeTok}[1]{\textcolor[rgb]{0.77,0.63,0.00}{#1}}
\newcommand{\BaseNTok}[1]{\textcolor[rgb]{0.00,0.00,0.81}{#1}}
\newcommand{\BuiltInTok}[1]{#1}
\newcommand{\CharTok}[1]{\textcolor[rgb]{0.31,0.60,0.02}{#1}}
\newcommand{\CommentTok}[1]{\textcolor[rgb]{0.56,0.35,0.01}{\textit{#1}}}
\newcommand{\CommentVarTok}[1]{\textcolor[rgb]{0.56,0.35,0.01}{\textbf{\textit{#1}}}}
\newcommand{\ConstantTok}[1]{\textcolor[rgb]{0.00,0.00,0.00}{#1}}
\newcommand{\ControlFlowTok}[1]{\textcolor[rgb]{0.13,0.29,0.53}{\textbf{#1}}}
\newcommand{\DataTypeTok}[1]{\textcolor[rgb]{0.13,0.29,0.53}{#1}}
\newcommand{\DecValTok}[1]{\textcolor[rgb]{0.00,0.00,0.81}{#1}}
\newcommand{\DocumentationTok}[1]{\textcolor[rgb]{0.56,0.35,0.01}{\textbf{\textit{#1}}}}
\newcommand{\ErrorTok}[1]{\textcolor[rgb]{0.64,0.00,0.00}{\textbf{#1}}}
\newcommand{\ExtensionTok}[1]{#1}
\newcommand{\FloatTok}[1]{\textcolor[rgb]{0.00,0.00,0.81}{#1}}
\newcommand{\FunctionTok}[1]{\textcolor[rgb]{0.00,0.00,0.00}{#1}}
\newcommand{\ImportTok}[1]{#1}
\newcommand{\InformationTok}[1]{\textcolor[rgb]{0.56,0.35,0.01}{\textbf{\textit{#1}}}}
\newcommand{\KeywordTok}[1]{\textcolor[rgb]{0.13,0.29,0.53}{\textbf{#1}}}
\newcommand{\NormalTok}[1]{#1}
\newcommand{\OperatorTok}[1]{\textcolor[rgb]{0.81,0.36,0.00}{\textbf{#1}}}
\newcommand{\OtherTok}[1]{\textcolor[rgb]{0.56,0.35,0.01}{#1}}
\newcommand{\PreprocessorTok}[1]{\textcolor[rgb]{0.56,0.35,0.01}{\textit{#1}}}
\newcommand{\RegionMarkerTok}[1]{#1}
\newcommand{\SpecialCharTok}[1]{\textcolor[rgb]{0.00,0.00,0.00}{#1}}
\newcommand{\SpecialStringTok}[1]{\textcolor[rgb]{0.31,0.60,0.02}{#1}}
\newcommand{\StringTok}[1]{\textcolor[rgb]{0.31,0.60,0.02}{#1}}
\newcommand{\VariableTok}[1]{\textcolor[rgb]{0.00,0.00,0.00}{#1}}
\newcommand{\VerbatimStringTok}[1]{\textcolor[rgb]{0.31,0.60,0.02}{#1}}
\newcommand{\WarningTok}[1]{\textcolor[rgb]{0.56,0.35,0.01}{\textbf{\textit{#1}}}}
\usepackage{longtable,booktabs,array}
\usepackage{calc} % for calculating minipage widths
% Correct order of tables after \paragraph or \subparagraph
\usepackage{etoolbox}
\makeatletter
\patchcmd\longtable{\par}{\if@noskipsec\mbox{}\fi\par}{}{}
\makeatother
% Allow footnotes in longtable head/foot
\IfFileExists{footnotehyper.sty}{\usepackage{footnotehyper}}{\usepackage{footnote}}
\makesavenoteenv{longtable}
\usepackage{graphicx}
\makeatletter
\def\maxwidth{\ifdim\Gin@nat@width>\linewidth\linewidth\else\Gin@nat@width\fi}
\def\maxheight{\ifdim\Gin@nat@height>\textheight\textheight\else\Gin@nat@height\fi}
\makeatother
% Scale images if necessary, so that they will not overflow the page
% margins by default, and it is still possible to overwrite the defaults
% using explicit options in \includegraphics[width, height, ...]{}
\setkeys{Gin}{width=\maxwidth,height=\maxheight,keepaspectratio}
% Set default figure placement to htbp
\makeatletter
\def\fps@figure{htbp}
\makeatother
\setlength{\emergencystretch}{3em} % prevent overfull lines
\providecommand{\tightlist}{%
  \setlength{\itemsep}{0pt}\setlength{\parskip}{0pt}}
\setcounter{secnumdepth}{-\maxdimen} % remove section numbering
\ifluatex
  \usepackage{selnolig}  % disable illegal ligatures
\fi

\title{Atividade trilha 1 - Analise estátisticaca}
\author{}
\date{\vspace{-2.5em}}

\begin{document}
\maketitle

Importação da planilha PelicanStores

\begin{Shaded}
\begin{Highlighting}[]
\FunctionTok{library}\NormalTok{(readxl)}
\NormalTok{PelicanStores }\OtherTok{\textless{}{-}} \FunctionTok{read\_excel}\NormalTok{(}\StringTok{"C:/Users/guilherme.ssantos/Documents/R/AnaliseEstatisticaWithR/bases/PelicanStores.xlsx"}\NormalTok{, }
    \AttributeTok{col\_types =} \FunctionTok{c}\NormalTok{(}\StringTok{"numeric"}\NormalTok{, }\StringTok{"text"}\NormalTok{, }\StringTok{"numeric"}\NormalTok{, }
        \StringTok{"numeric"}\NormalTok{, }\StringTok{"text"}\NormalTok{, }\StringTok{"text"}\NormalTok{, }\StringTok{"text"}\NormalTok{, }
        \StringTok{"numeric"}\NormalTok{))}
\end{Highlighting}
\end{Shaded}

Construindo dicionário de dados

\begin{Shaded}
\begin{Highlighting}[]
\NormalTok{dicionarioDeDadosPelicanStores }\OtherTok{\textless{}{-}} \FunctionTok{data.frame}\NormalTok{(}
                   \AttributeTok{NomeDaVariavel =} \FunctionTok{c}\NormalTok{(}\StringTok{"Cliente"}\NormalTok{, }\StringTok{"Tipo de Cliente"}\NormalTok{, }\StringTok{"Itens"}\NormalTok{,}
                                \StringTok{"Vendas líquidas"}\NormalTok{, }\StringTok{"Método de pagamento"}\NormalTok{, }\StringTok{"Gênero"}\NormalTok{,}
                                \StringTok{"Estado civil"}\NormalTok{, }\StringTok{"Idade"}\NormalTok{),}
\NormalTok{                   Descrição }\OtherTok{=} \FunctionTok{c}\NormalTok{(}\StringTok{"Id dos clientes"}\NormalTok{,}\StringTok{"Categoria dos clientes"}\NormalTok{,}\StringTok{"Quantidade de itens"}\NormalTok{,}\StringTok{" Valor da venda liquida"}\NormalTok{,}\StringTok{"Bandeira do cartão/cartão póprio"}\NormalTok{,}
                                 \StringTok{"Categoria de genero do cliente"}\NormalTok{,}\StringTok{"Estado civil do cliente"}\NormalTok{,}\StringTok{"Idade do cliente"}\NormalTok{),}
\NormalTok{                   TipoDaVariável }\OtherTok{=} \FunctionTok{c}\NormalTok{(}\StringTok{"Categorica"}\NormalTok{,}\StringTok{"Categorica"}\NormalTok{,}\StringTok{"Numerica"}\NormalTok{,}\StringTok{"Numerica"}\NormalTok{,}\StringTok{"Categorica"}\NormalTok{,}\StringTok{"Categorica"}\NormalTok{,}\StringTok{"Categorica"}\NormalTok{,}\StringTok{"Numerica"}\NormalTok{),}
                   
\NormalTok{                   TipoDeMensuração }\OtherTok{=} \FunctionTok{c}\NormalTok{(}\StringTok{"nominal"}\NormalTok{,}\StringTok{"nominal"}\NormalTok{,}\StringTok{"discreta"}\NormalTok{,}\StringTok{"razão"}\NormalTok{,}\StringTok{"nominal"}\NormalTok{,}\StringTok{"nominal"}\NormalTok{,}\StringTok{"nominal"}\NormalTok{,}\StringTok{"razão"}\NormalTok{),}
                   
                   \AttributeTok{ValoresPossiveis =} \FunctionTok{c}\NormalTok{(}\StringTok{"Identificação dos clientes"}\NormalTok{,}\StringTok{"Regular/Promossional"}\NormalTok{,}\StringTok{"Numeros inteiros positivos"}\NormalTok{,}\StringTok{"Numeros reais positivos"}\NormalTok{,}\StringTok{"Métodos de pagamentos"}\NormalTok{, }\StringTok{" Masculino/Feminino"}\NormalTok{, }\StringTok{" Casado/solteiro"}\NormalTok{,}\StringTok{"média, variação percentual"}\NormalTok{)}
                   
                   
\NormalTok{                   )}
\end{Highlighting}
\end{Shaded}

Dicionario de dados:

\begin{Shaded}
\begin{Highlighting}[]
\FunctionTok{library}\NormalTok{(knitr)}
\FunctionTok{kable}\NormalTok{(dicionarioDeDadosPelicanStores)}
\end{Highlighting}
\end{Shaded}

\begin{longtable}[]{@{}
  >{\raggedright\arraybackslash}p{(\columnwidth - 8\tabcolsep) * \real{0.18}}
  >{\raggedright\arraybackslash}p{(\columnwidth - 8\tabcolsep) * \real{0.29}}
  >{\raggedright\arraybackslash}p{(\columnwidth - 8\tabcolsep) * \real{0.13}}
  >{\raggedright\arraybackslash}p{(\columnwidth - 8\tabcolsep) * \real{0.15}}
  >{\raggedright\arraybackslash}p{(\columnwidth - 8\tabcolsep) * \real{0.24}}@{}}
\toprule
NomeDaVariavel & Descrição & TipoDaVariável & TipoDeMensuração &
ValoresPossiveis \\
\midrule
\endhead
Cliente & Id dos clientes & Categorica & nominal & Identificação dos
clientes \\
Tipo de Cliente & Categoria dos clientes & Categorica & nominal &
Regular/Promossional \\
Itens & Quantidade de itens & Numerica & discreta & Numeros inteiros
positivos \\
Vendas líquidas & Valor da venda liquida & Numerica & razão & Numeros
reais positivos \\
Método de pagamento & Bandeira do cartão/cartão póprio & Categorica &
nominal & Métodos de pagamentos \\
Gênero & Categoria de genero do cliente & Categorica & nominal &
Masculino/Feminino \\
Estado civil & Estado civil do cliente & Categorica & nominal &
Casado/solteiro \\
Idade & Idade do cliente & Numerica & razão & média, variação
percentual \\
\bottomrule
\end{longtable}

\end{document}
